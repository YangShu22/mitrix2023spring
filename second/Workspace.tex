\documentclass[UTF8]{uofa-eng-assignment}

\usepackage{lipsum}
\usepackage{CTEX}
\newcommand*{\name}{\kaishu  杨澍}
\newcommand*{\id}{202218019427012}
\newcommand*{\course}{\kaishu 矩阵论}
\newcommand*{\assignment}{First Assignment}

\begin{document}

\maketitle

\begin{enumerate}

%%%%%%%%%%%%%%%%%%%%
\item \textbf{Question 2.2.2:}
%%%%%%%%%%%%%%%%%%%%

\textbf{Solve:}
因为
$$Ax=\lambda x$$
所以有
$$A^mx=\lambda^m x$$
即有
$$
\begin{aligned}
||A^mx||&=||\lambda^m x||\\
&=|\lambda|^m||x||
\end{aligned}
$$
且有
$||A^mx||\leq||A^m|| ||x||$
所以
$$
|\lambda|\leq \sqrt[m]{||A^m||}
$$
%%%%%%%%%%%%%%%%%%%%
\item \textbf{Question 2.2.5:}
%%%%%%%%%%%%%%%%%%%%

\textbf{Solve}
$$
\begin{aligned}
||A||_S&=\max \frac{||Ax||_S}{||x||_S}\\
&=\max \frac{||SAx||_2}{||Sx||_2}\\
&=\max \frac{||SAS^{-1}y||_2}{||y||_2}\\
&=||SAS^{-1}||_2
\end{aligned}
$$

%%%%%%%%%%%%%%%%%%%%
\item \textbf{Question 2.2.7:}
%%%%%%%%%%%%%%%%%%%%
\textbf{Solve}
先计算基2到基1的过渡矩阵
$$C^{-1}=\left[
    \begin{matrix}
        -1&0&1\\
        1&1&0\\
        1&2&1\\
    \end{matrix}\right]
$$
由此可以计算T在基2下的矩阵为
$$
B=C^{-1}AC=\left[
    \begin{matrix}
        -1&1&-2\\
        2&2&0\\
        3&0&2\\
    \end{matrix}\right]
$$
%%%%%%%%%%%%%%%%%%%%
\item \textbf{Question 1.2.14:}
%%%%%%%%%%%%%%%%%%%%
\textbf{Solve}
矩阵A的特征多项式为
$$det(\lambda I-A)=(\lambda-1 )(\lambda+1)(\lambda)-(\lambda-1 )=\lambda^3-2\lambda+1=0
$$
所以
$$A^3-2A+1=0$$
$$
\begin{aligned}
    &2A^9-3A^5+A^4+A^2-4I\\
    &=24A^2-37A+10I\\
    &=\left[
        \begin{matrix}
            -3&48&-26\\
            0&95&-61\\
            0&-61&34\\
        \end{matrix}\right]
\end{aligned}    
$$
%%%%%%%%%%%%%%%%%%%%
\item \textbf{Question 1.2.16:}
%%%%%%%%%%%%%%%%%%%%
\textbf{Solve}
矩阵A的特征多项式为
$$det(\lambda I-A)=(\lambda-9)(\lambda+9)^2=\lambda^3+9\lambda^2-81\lambda-729$$
由H-K定理得最小多项式为
$$(\lambda-9)(\lambda+9)$$
%%%%%%%%%%%%%%%%%%%%
\item \textbf{Question 1.2.18:}
%%%%%%%%%%%%%%%%%%%%
\textbf{Solve}
因为$\lambda_0$为$T_1$特征值所以
$$T_1x=\lambda_0 x $$
$$
\begin{aligned}
    T_1T_2x&=T_2 \lambda_0 x\\
    &=\lambda_0(T_2x)
\end{aligned}
$$
即$T_2x \in V_\lambda0$
所以$V_\lambda0$为$T_2$不变子空间
%%%%%%%%%%%%%%%%%%%%
\item \textbf{Question 1.2.19:}
%%%%%%%%%%%%%%%%%%%%
\textbf{Solve}
$$det(\lambda I-A)=(\lambda-1)(\lambda-2)(\lambda+1)$$
所以若当标准型为
$$
\left[
        \begin{matrix}
            1&0&0\\
            0&2&0\\
            0&0&-1\\
        \end{matrix}\right]
$$
%%%%%%%%%%%%%%%%%%%%
\item \textbf{Question 1.3.7:}
%%%%%%%%%%%%%%%%%%%%
\textbf{Solve}
先证充分性,若$x_1......x_n$线性相关,则有
$$\sum{k_ix_i}=0$$
使之与$x_m$做内积得
$$\sum{k_i(x_i,x_m)}=0$$
故detB=0
再证必要性,
由detB!=0可知,
$$\sum{k_i(x_i,x_m)}=0$$
即$x_1......x_n$线性无关。
%%%%%%%%%%%%%%%%%%%%
\item \textbf{Question 1.3.9:}
%%%%%%%%%%%%%%%%%%%%
$$
\begin{aligned}
    (Tx,Tx)&=(x-2(y,x)y,x-2(y,x)y)\\
    &=(x,x)-4(y,x)(x,y)+4(y,x)(y,x)(y,y)\\
    &=(x,x)
\end{aligned}
$$
所以为正交变换。
%%%%%%%%%%%%%%%%%%%%
\item \textbf{Question 1.3.11:}
%%%%%%%%%%%%%%%%%%%%
\textbf{Solve}
先求得该矩阵的特征值
$$det(\lambda I-A)=0$$
解得$$\lambda_1=0,\lambda_2=\sqrt{2},\lambda_3=-\sqrt{2}$$
对应的特征向量为
$$
x_1=(0,\frac{i}{\sqrt{2}},\frac{1}{\sqrt{2}})^T,
x_2=(\frac{1}{\sqrt{2}},-\frac{i}{2},\frac{1}{2} )^T,
x_3=(-\frac{1}{\sqrt{2}},-\frac{i}{2},\frac{1}{2} )^T 
$$
所以可求得
$$P=(x_1,x_2,x3)$$
%%%%%%%%%%%%%%%%%%%%
\item \textbf{Question 1.3.15:}
%%%%%%%%%%%%%%%%%%%%
\textbf{Solve}
(1)其中的一个标准正交基为
$$
X_1=\left[
        \begin{matrix}
            1&0\\
            0&1\\
        \end{matrix}\right],
X_2=\left[
        \begin{matrix}
            0&1\\
            1&0\\
        \end{matrix}\right],
$$
且$X_1\odot X_2=0$所以这是一对正交基,将他们标准化得标准正交基为
$$
a_1=\left[
        \begin{matrix}
            \frac{1}{\sqrt{2}}&0\\
            0&\frac{1}{\sqrt{2}}\\
        \end{matrix}\right],
a_2=\left[
        \begin{matrix}
            0&\frac{1}{\sqrt{2}}\\
            \frac{1}{\sqrt{2}}&0\\
        \end{matrix}\right],
$$
(2)
$$
\begin{aligned}
    \left\langle TX,Y \right\rangle&=\left\langle XP+X^T,Y \right\rangle\\
    &=\left\langle XP,Y \right\rangle+\left\langle X^T,Y \right\rangle
    &=\left\langle X,Y \right\rangle P+\left\langle X,Y^T \right\rangle
    = \left\langle X,TY \right\rangle
\end{aligned}
$$
所以为对称变换。


(3)
$$T[a_1,a_2]=[a_1,a_2]A$$
由此可以求得T在基$(a_1,a_2)$下的矩阵
$$
A=\left[
        \begin{matrix}
            1&1\\
            1&1\\
\end{matrix}\right]
$$
设C为基$(a_1,a_2)$到目标基$(b_1,b_2)$的过渡矩阵
则对所要求的基下T对应的矩阵B应满足
$$B=C^{-1}AC$$
即
$$A=CBC^{-1}$$且B为对角阵,即A可进行特征值分解得到B
解得
$$
\lambda_1=2,\lambda_2=0
$$
$$
C=\left[
        \begin{matrix}
            \frac{1}{\sqrt{2}}&\frac{1}{\sqrt{2}}\\
            \frac{1}{\sqrt{2}}&-\frac{1}{\sqrt{2}}\\
\end{matrix}\right]
$$
由此可以得到
$$
b_1=\left[
        \begin{matrix}
            \frac{1}{2}&\frac{1}{2}\\
            \frac{1}{2}&\frac{1}{2}\\
        \end{matrix}\right],
b_2=\left[
        \begin{matrix}
            \frac{1}{2}&-\frac{1}{2}\\
            -\frac{1}{2}&\frac{1}{2}\\
        \end{matrix}\right],
$$
\end{enumerate}
\end{document}
