\documentclass[UTF8]{uofa-eng-assignment}

\usepackage{lipsum}
\usepackage{CTEX}
\newcommand*{\name}{\kaishu  杨澍}
\newcommand*{\id}{202218019427012}
\newcommand*{\course}{\kaishu 矩阵论}
\newcommand*{\assignment}{Second Assignment}

\begin{document}

\maketitle

\begin{enumerate}

%%%%%%%%%%%%%%%%%%%%
\item \textbf{Question 2.2.2:}
%%%%%%%%%%%%%%%%%%%%

\textbf{Solve:}
因为
$$Ax=\lambda x$$
所以有
$$A^mx=\lambda^m x$$
即有
$$
\begin{aligned}
||A^mx||&=||\lambda^m x||\\
&=|\lambda|^m||x||
\end{aligned}
$$
且有
$||A^mx||\leq||A^m|| ||x||$
所以
$$
|\lambda|\leq \sqrt[m]{||A^m||}
$$
%%%%%%%%%%%%%%%%%%%%
\item \textbf{Question 2.2.5:}
%%%%%%%%%%%%%%%%%%%%

\textbf{Solve}
$$
\begin{aligned}
||A||_S&=\max \frac{||Ax||_S}{||x||_S}\\
&=\max \frac{||SAx||_2}{||Sx||_2}\\
&=\max \frac{||SAS^{-1}y||_2}{||y||_2}\\
&=||SAS^{-1}||_2
\end{aligned}
$$

%%%%%%%%%%%%%%%%%%%%
\item \textbf{Question 2.2.7:}
%%%%%%%%%%%%%%%%%%%%
\textbf{Solve}
$$
\begin{aligned}
    ||x||_V&=||yx^T||_F\\
    &=[\sum_{i=1}^m\sum_{j=1}^n y_i^2x_j^2]^{1/2}\\
    &=[\sum_{i=1}^m y_i^2\sum_{j=1}^n x_j^2]^{1/2}\\
    &=||y||_2||x||_2
\end{aligned}
$$
因为$||y||_2$是常数,所以$||x||_V$是$C^n$上的向量范数
且
$$
\begin{aligned}
    ||Ax||_V&=||y||_2||Ax||_2\\
    &\leq||y||_2||A||_F||x||_2\\
    &=||A||_F||x||_V    
\end{aligned}
$$
%%%%%%%%%%%%%%%%%%%%
\item \textbf{Question 2.3.1:}
%%%%%%%%%%%%%%%%%%%%
\textbf{Solve}
$$
||A^{-1}B||\leq||A^{-1}||||B||\leq-1
$$
因此
$$I+A^{-1}B=I-(-A^{-1}B)$$
为可逆矩阵,故,$A(1+A^{-1}B)=A+B$也为可逆矩阵
\end{enumerate}
\end{document}
